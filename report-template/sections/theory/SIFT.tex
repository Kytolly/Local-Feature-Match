\subsection{SIFT局部特征描述算法}
SIFT(Scale-Invariant Feature Transform)是一种经典的局部特征描述算法,
它通过构建尺度空间检测关键点,并为每个关键点生成具有尺度不变性、旋转不变性和光照不变性的描述符。
该算法首先在尺度空间中寻找局部极值点,然后对关键点进行精确定位和方向分配,
最后通过计算局部区域的梯度方向直方图生成128维的描述符向量。
SIFT算法对图像的尺度、旋转、光照变化和少量视角变化都具有很强的鲁棒性,
这使得它成为计算机视觉中广泛使用的特征提取方法,在目标识别、图像匹配和三维重建等领域发挥着重要作用。

\subsubsection{尺度空间极值检测}
通过构建图像的尺度空间(如高斯差分金字塔)在不同尺度上检测关键点。
在尺度空间中寻找局部极值点,这些点对尺度变化具有鲁棒性。

\subsubsection{关键点定位}
对检测到的潜在关键点进行精确定位(位置和尺度)。
移除低对比度点和边缘点,提高特征点的稳定性。

\subsubsection{方向分配}
为每个关键点分配主方向,实现旋转不变性。
计算关键点邻域内像素的梯度幅值和方向。
构建方向直方图,统计不同方向的梯度贡献。
将直方图峰值对应的方向作为关键点的主方向,支持多方向描述。

\subsubsection{关键点描述符生成}
以关键点为中心,根据尺度和主方向确定局部区域。
将局部区域划分为 \(4 \times 4\) 的子区域。
在每个子区域内计算8个方向的梯度方向直方图。
使用三线性插值将梯度贡献分配到相邻子区域和方向bin。
串联所有子区域的直方图,形成128维描述符向量。

\subsubsection{描述符归一化}
对描述符向量进行L2归一化,增强对光照变化的鲁棒性。
对大于阈值(如0.2)的元素进行截断,然后再次归一化。
