
\subsection{NNDR算法}
NNDR(Nearest Neighbor Distance Ratio),
即最近邻距离比算法,是一种常用于特征匹配中的方法,
特别是在使用像 SIFT、SURF 等描述符进行特征点匹配时。
它的主要目的是为了提高匹配的准确性,过滤掉那些不确定或错误的匹配对。

\subsubsection{基本思想}
一个好的特征匹配对,其描述符在特征空间中的距离应该远小于它与次近邻描述符的距离。
换句话说,一个特征点在图像 A 中的最佳匹配点在图像 B 中应该是“独一无二”的最近邻,
而不是与多个点都非常接近。
如果一个特征点在图像 A 中的描述符与图像 B 中的两个或多个描述符都非常接近,
那么这可能表明这个特征点不够独特,或者存在歧义,
这种匹配对很可能是错误的(即假阳性)。
通过比较最近邻和次近邻的距离,我们可以量化这种“独特性”或“歧义性”。
NNDR 算法通常作为特征匹配流水线中的一个后处理步骤,紧随在特征描述符计算和初步的最近邻搜索之后。
它的输出是一组被认为可靠的特征匹配对,这些匹配对可以进一步用于估计图像间的几何变换
(如单应性矩阵或基础矩阵),进行图像拼接、目标跟踪或三维重建等任务。
总之,NNDR 是一种简单而有效的特征匹配过滤方法,通过比较最近邻和次近邻的距离,
能够显著提高匹配的准确性,减少误匹配的数量。
