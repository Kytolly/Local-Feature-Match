\section{代码示例}

核心代码如代码1所示。

\begin{lstlisting}[caption={student.py}, label={lst:code-example}, captionpos=t, language=python]
  import numpy as np
  import matplotlib.pyplot as plt
  from skimage import feature, img_as_int
  from skimage.measure import regionprops
  import cv2
  from skimage.filters import gaussian, sobel_h, sobel_v
  
  
  def get_interest_points(image, feature_width):
      '''
      Returns a set of interest points for the input image
  
      (Please note that we recommend implementing this function last and using cheat_interest_points()
      to test your implementation of get_features() and match_features())
  
      Implement the Harris corner detector (See Szeliski 4.1.1) to start with.
      You do not need to worry about scale invariance or keypoint orientation estimation
      for your Harris corner detector.
      You can create additional interest point detector functions (e.g. MSER)
      for extra credit.
  
      If you're finding spurious (false/fake) interest point detections near the boundaries,
      it is safe to simply suppress the gradients / corners near the edges of
      the image.
  
      Useful functions: A working solution does not require the use of all of these
      functions, but depending on your implementation, you may find some useful. Please
      reference the documentation for each function/library and feel free to come to hours
      or post on Piazza with any questions
  
          - skimage.feature.peak_local_max
          - skimage.measure.regionprops
  
  
      :params:
      :image: a grayscale or color image (your choice depending on your implementation)
      :feature_width:
  
      :returns:
      :xs: an np array of the x coordinates of the interest points in the image
      :ys: an np array of the y coordinates of the interest points in the image
  
      :optional returns (may be useful for extra credit portions):
      :confidences: an np array indicating the confidence (strength) of each interest point
      :scale: an np array indicating the scale of each interest point
      :orientation: an np array indicating the orientation of each interest point
  
      '''
  
      # TODO: Your implementation here!
      if image.ndim == 3:
          image = cv2.cvtColor(image, cv2.COLOR_BGR2GRAY)
  
      image = img_as_int(image)
  
      Ix = sobel_v(image)
      Iy = sobel_h(image)
      Ix2 = Ix * Ix
      Iy2 = Iy * Iy
      Ixy = Ix * Iy
      
      k = 0.00001 
      sigma = feature_width / 6.0 
      min_distance = int(feature_width / 3) 
      threshold_rate = 0.0005 
      
  
      Sxx = gaussian(Ix2, sigma=sigma)
      Syy = gaussian(Iy2, sigma=sigma)
      Sxy = gaussian(Ixy, sigma=sigma)
  
      det_M = Sxx * Syy - Sxy * Sxy
      trace_M = Sxx + Syy
      harris_response = det_M - k * (trace_M ** 2)
      threshold = threshold_rate * harris_response.max()
      
      coords = feature.peak_local_max(harris_response, min_distance=min_distance, threshold_abs=threshold)
      ys, xs = coords[:, 0], coords[:, 1] # peak_local_max (row, col), (y, x)
  
      border_margin = int(feature_width / 2)
      height, width = image.shape
      valid_indices = np.where(
          (xs >= border_margin) & (xs < width - border_margin) &
          (ys >= border_margin) & (ys < height - border_margin)
      )
      xs = xs[valid_indices]
      ys = ys[valid_indices]
  
      # xs = np.asarray([0])
      # ys = np.asarray([0])
      return xs, ys
  
  
  def get_features(image, x, y, feature_width):
      '''
      Returns a set of feature descriptors for a given set of interest points.
  
      (Please note that we reccomend implementing this function after you have implemented
      match_features)
  
      To start with, you might want to simply use normalized patches as your
      local feature. This is very simple to code and works OK. However, to get
      full credit you will need to implement the more effective SIFT-like descriptor
      (See Szeliski 4.1.2 or the original publications at
      http://www.cs.ubc.ca/~lowe/keypoints/)
  
      Your implementation does not need to exactly match the SIFT reference.
      Here are the key properties your (baseline) descriptor should have:
      (1) a 4x4 grid of cells, each descriptor_window_image_width/4.
      (2) each cell should have a histogram of the local distribution of
          gradients in 8 orientations. Appending these histograms together will
          give you 4x4 x 8 = 128 dimensions.
      (3) Each feature should be normalized to unit length
  
      You do not need to perform the interpolation in which each gradient
      measurement contributes to multiple orientation bins in multiple cells
      As described in Szeliski, a single gradient measurement creates a
      weighted contribution to the 4 nearest cells and the 2 nearest
      orientation bins within each cell, for 8 total contributions. This type
      of interpolation probably will help, though.
  
      You do not need to do the normalize -> threshold -> normalize again
      operation as detailed in Szeliski and the SIFT paper. It can help, though.
  
      Another simple trick which can help is to raise each element of the final
      feature vector to some power that is less than one.
  
      Useful functions: A working solution does not require the use of all of these
      functions, but depending on your implementation, you may find some useful. Please
      reference the documentation for each function/library and feel free to come to hours
      or post on Piazza with any questions
  
          - skimage.filters (library)
  
  
      :params:
      :image: a grayscale or color image (your choice depending on your implementation)
      :x: np array of x coordinates of interest points
      :y: np array of y coordinates of interest points
      :feature_width: in pixels, is the local feature width. You can assume
                      that feature_width will be a multiple of 4 (i.e. every cell of your
                      local SIFT-like feature will have an integer width and height).
      If you want to detect and describe features at multiple scales or
      particular orientations you can add input arguments.
  
      :returns:
      :features: np array of computed features. It should be of size
              [len(x) * feature dimensionality] (for standard SIFT feature
              dimensionality is 128)
  
      '''
  
      # TODO: Your implementation here!
  
      # This is a placeholder - replace this with your features!
      # features = np.asarray([0])
  
      num_interest_points = len(x)
      features = np.zeros((num_interest_points, 128)) # 128 dimensions for SIFT-like descriptor
  
      # Convert image to grayscale if it's color (although input is expected to be grayscale based on notebook)
      if image.ndim == 3:
          image = cv2.cvtColor(image, cv2.COLOR_BGR2GRAY)
  
      # Convert image to float for gradient calculations
      image = img_as_int(image)
  
      # Calculate gradients
      # Using Sobel filters again for consistency with get_interest_points
      Ix = sobel_v(image)
      Iy = sobel_h(image)
  
      # Calculate gradient magnitude and orientation
      magnitude = np.sqrt(Ix**2 + Iy**2)
      # Add a small epsilon to avoid division by zero in arctan2 if both Ix and Iy are 0
      orientation = np.arctan2(Iy, Ix + 1e-6) * 180 / np.pi # Convert to degrees
      # Normalize orientation to be within [0, 360)
      orientation = (orientation + 360) % 360
  
      # Define cell and bin parameters
      num_cells = 4
      num_bins = 8
      cell_width = feature_width // num_cells
      bin_size = 360 // num_bins # Size of each orientation bin in degrees
  
      # Process each interest point
      for i in range(num_interest_points):
          # Get the coordinates of the current interest point
          px, py = x[i], y[i]
  
          # Define the bounding box for the feature window
          # Need to be careful with boundary conditions
          half_width = feature_width // 2
          min_x = max(0, px - half_width)
          max_x = min(image.shape[1] - 1, px + half_width -1) # -1 because max_x is inclusive index
          min_y = max(0, py - half_width)
          max_y = min(image.shape[0] - 1, py + half_width -1) # -1 because max_y is inclusive index
  
          # Ensure window size is correct even at boundaries
          # This might require padding the image or adjusting the window size/handling points near border
          # For simplicity here, we'll just take the available window and handle potential smaller size
          # A more robust implementation might pad the image.
  
          # Extract the local patch
          # Note: Slicing includes start but excludes end. Adjust max indices accordingly.
          patch_mag = magnitude[min_y : max_y + 1, min_x : max_x + 1]
          patch_ori = orientation[min_y : max_y + 1, min_x : max_x + 1]
  
          # Ensure patch is exactly feature_width x feature_width. If not, there's likely an issue with
          # how interest points near the border were handled or how min/max were calculated.
          # Given the border suppression in get_interest_points, this window *should* be full size
          # if border_margin >= feature_width / 2.
          # For now, proceed assuming full size or handle potential smaller size in histogramming.
  
          descriptor = np.zeros(num_cells * num_cells * num_bins) # 4x4 grid * 8 bins = 128
  
          # Build histograms for each cell
          # Instead of iterating through cells then pixels, iterate through pixels and contribute to cells/bins
          # We need patch-relative coordinates for interpolation
          patch_height, patch_width = patch_mag.shape
  
          # Define Gaussian weighting window
          # Sigma for Gaussian is typically half the window size
          sigma_spatial = feature_width / 2.0
          y_coords, x_coords = np.indices(patch_mag.shape)
          # Calculate distance from the center of the patch (which aligns with the interest point)
          # Center of patch is at (half_width - 0.5, half_width - 0.5) if patch is feature_width x feature_width
          patch_center_r = patch_height / 2.0 - 0.5
          patch_center_c = patch_width / 2.0 - 0.5
          distances_sq = (x_coords - patch_center_c)**2 + (y_coords - patch_center_r)**2
          gaussian_weights_patch = np.exp(-distances_sq / (2 * sigma_spatial**2))
  
          # Process each pixel in the patch
          # Coordinates relative to the top-left of the patch (0 to feature_width-1)
          for r in range(patch_height):
              for c in range(patch_width):
                  mag = patch_mag[r, c]
                  ori = patch_ori[r, c]
                  gaussian_weight = gaussian_weights_patch[r, c]
  
                  # Apply Gaussian weight to the magnitude
                  weighted_mag = mag * gaussian_weight
  
                  # Normalize orientation to be within [0, 360) for bin calculation
                  ori = (ori + 360) % 360
  
                  # Calculate the float bin index (0 to num_bins - epsilon)
                  float_bin_index = ori / bin_size
  
                  # Get the two nearest integer bin indices
                  bin1_index = int(np.floor(float_bin_index)) % num_bins
                  bin2_index = int(np.ceil(float_bin_index)) % num_bins
  
                  # Calculate orientation weights for the two bins
                  # Fractional part of the float bin index gives the position between the two bins
                  fractional_part_ori = float_bin_index - np.floor(float_bin_index)
                  ori_weight1 = (1 - fractional_part_ori) # Weight for bin1_index
                  ori_weight2 = fractional_part_ori     # Weight for bin2_index
  
                  # Calculate spatial position relative to the top-left of the feature window (0 to feature_width-1)
                  # and then map to cell coordinates (0 to num_cells)
                  # Pixel (r, c) in patch is at (c, r) relative to patch top-left
                  # Map patch coordinates to cell coordinates. Cell centers are at (0.5, 1.5, 2.5, 3.5) * cell_width + cell_width/2
                  # Pixel (c, r) in patch falls into a cell with top-left at floor(c/cell_width)*cell_width, floor(r/cell_width)*cell_width
  
                  # Calculate float cell coordinates based on pixel position within the feature window (0 to num_cells)
                  # Pixel (c, r) is at spatial location (c+0.5, r+0.5) within the patch (0 to feature_width)
                  float_cell_x = (c + 0.5) / cell_width
                  float_cell_y = (r + 0.5) / cell_width
  
                  # Get the two nearest integer cell indices in each dimension (0 to num_cells)
                  cell1_x = int(np.floor(float_cell_x))
                  cell2_x = int(np.ceil(float_cell_x))
                  cell1_y = int(np.floor(float_cell_y))
                  cell2_y = int(np.ceil(float_cell_y))
  
                  # Calculate spatial weights for the four cells (bilinear interpolation)
                  fractional_part_x = float_cell_x - np.floor(float_cell_x)
                  fractional_part_y = float_cell_y - np.floor(float_cell_y)
  
                  # Weights for the 4 target cells based on bilinear interpolation
                  spatial_weight11 = (1 - fractional_part_x) * (1 - fractional_part_y) # cell1_x, cell1_y
                  spatial_weight21 = fractional_part_x * (1 - fractional_part_y)     # cell2_x, cell1_y
                  spatial_weight12 = (1 - fractional_part_x) * fractional_part_y     # cell1_x, cell2_y
                  spatial_weight22 = fractional_part_x * fractional_part_y         # cell2_x, cell2_y
  
                  # Add weighted magnitude to the corresponding bins in the descriptor
                  # Iterate over the 4 target cells and 2 target orientation bins
                  target_contributions = [
                      (cell1_x, cell1_y, spatial_weight11, bin1_index, ori_weight1),
                      (cell1_x, cell1_y, spatial_weight11, bin2_index, ori_weight2),
                      (cell2_x, cell1_y, spatial_weight21, bin1_index, ori_weight1),
                      (cell2_x, cell1_y, spatial_weight21, bin2_index, ori_weight2),
                      (cell1_x, cell2_y, spatial_weight12, bin1_index, ori_weight1),
                      (cell1_x, cell2_y, spatial_weight12, bin2_index, ori_weight2),
                      (cell2_x, cell2_y, spatial_weight22, bin1_index, ori_weight1),
                      (cell2_x, cell2_y, spatial_weight22, bin2_index, ori_weight2),
                  ]
  
                  for cur_cx, cur_cy, spatial_w, bin_idx, ori_w in target_contributions:
                      # Ensure cell indices are within the 0-3 range
                      if 0 <= cur_cx < num_cells and 0 <= cur_cy < num_cells:
                           # Calculate the final weighted magnitude contribution
                           final_weighted_mag = weighted_mag * spatial_w * ori_w
  
                           # Calculate the index in the flattened descriptor
                           descriptor_index = (cur_cy * num_cells + cur_cx) * num_bins + bin_idx
  
                           # Add to the descriptor
                           descriptor[descriptor_index] += final_weighted_mag
  
  
          # Normalize the descriptor to unit length
          # Add a small epsilon to the norm to prevent division by zero for uniform patches
          norm = np.linalg.norm(descriptor) + 1e-6
          descriptor /= norm
  
          # Apply thresholding and re-normalization (SIFT specific)
          threshold_value = 0.2
          descriptor[descriptor > threshold_value] = threshold_value
  
          # Second normalization after thresholding
          norm = np.linalg.norm(descriptor) + 1e-6
          descriptor /= norm
  
          # Assign the computed descriptor to the features array
          features[i, :] = descriptor
  
  
      return features
  
  
  def match_features(im1_features, im2_features):
      '''
      Implements the Nearest Neighbor Distance Ratio Test to assign matches between interest points
      in two images.
  
      Please implement the "Nearest Neighbor Distance Ratio (NNDR) Test" ,
      Equation 4.18 in Section 4.1.3 of Szeliski.
  
      For extra credit you can implement spatial verification of matches.
  
      Please assign a confidence, else the evaluation function will not work. Remember that
      the NNDR test will return a number close to 1 for feature points with similar distances.
      Think about how confidence relates to NNDR.
  
      This function does not need to be symmetric (e.g., it can produce
      different numbers of matches depending on the order of the arguments).
  
      A match is between a feature in im1_features and a feature in im2_features. We can
      represent this match as a the index of the feature in im1_features and the index
      of the feature in im2_features
  
      Useful functions: A working solution does not require the use of all of these
      functions, but depending on your implementation, you may find some useful. Please
      reference the documentation for each function/library and feel free to come to hours
      or post on Piazza with any questions
  
          - zip (python built in function)
  
      :params:
      :im1_features: an np array of features returned from get_features() for interest points in image1
      :im2_features: an np array of features returned from get_features() for interest points in image2
  
      :returns:
      :matches: an np array of dimension k x 2 where k is the number of matches. The first
              column is an index into im1_features and the second column is an index into im2_features
      :confidences: an np array with a real valued confidence for each match
      '''
  
      # TODO: Your implementation here!
  
      # These are placeholders - replace with your matches and confidences!
  
      matches = []
      confidences = []
  
      # Iterate through each feature in the first image
      for i, feature1 in enumerate(im1_features):
          # Calculate distances to all features in the second image
          distances = np.linalg.norm(im2_features - feature1, axis=1)
  
          # Find the indices of the two smallest distances
          # Use argpartition to find the indices of the k smallest elements efficiently
          if len(distances) < 2:
               # Need at least two features in im2 to calculate ratio
               continue
  
          # Get indices of the two smallest distances
          nearest_indices = np.argpartition(distances, 1)[:2]
  
          # Ensure we get the two actual smallest distances in correct order
          if distances[nearest_indices[0]] > distances[nearest_indices[1]]:
              nearest_indices = nearest_indices[[1, 0]] # Swap if necessary
  
          d1 = distances[nearest_indices[0]] # Distance to nearest neighbor
          d2 = distances[nearest_indices[1]] # Distance to second nearest neighbor
  
          # Apply Nearest Neighbor Distance Ratio (NNDR) test
          # A small ratio indicates a good match
          if d2 > 0: # Avoid division by zero
              ratio = d1 / d2
          else:
              # If second nearest neighbor has 0 distance, it's likely an identical feature
              # Treat this as a very confident match (ratio approaches 0)
              ratio = 0
  
          # Set a threshold for the ratio. This value may need tuning.
          # A common starting point is 0.8
          nndr_threshold = 0.8
  
          if ratio < nndr_threshold:
              # This is considered a valid match
              matches.append([i, nearest_indices[0]])
  
              # Calculate confidence (e.g., 1 - ratio, or based on ratio inverse)
              # Lower ratio means higher confidence
              confidence = 1.0 - ratio # Example confidence calculation
              confidences.append(confidence)
  
      # Convert lists to numpy arrays
      matches = np.asarray(matches)
      confidences = np.asarray(confidences)
  
      # Optional: Sort matches by confidence in descending order
      # (Useful for evaluation which might only look at top N matches)
      # sort_indices = np.argsort(confidences)[::-1]
      # matches = matches[sort_indices]
      # confidences = confidences[sort_indices]
  
      return matches, confidences
\end{lstlisting}
